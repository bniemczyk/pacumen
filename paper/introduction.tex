\section{Living Document}

This paper is a living document, please grab the latest version from \\
https://raw.githubusercontent.com/bniemczyk/pacumen/master/paper/pacumen.pdf

\section{Introduction}

The ability to dynamically classify and identify the entities behind
network traffic could help us with various network activities.  These
activities include IP network engineering with network provisioning
and network management. This ability is especially important for
surveillance for policy compliance and security.

Today's techniques, when applied to traffic that is not hidden rely on
the visibility of packet headers, port numbers and the protocol, and
stateful reconstruction of sessions.  These cues disappear when ssh,
SSL or other tunneling mechanisms are used to hide traffic. Whether
the hiding is legitimate or not, it is advantageous to identify
information about encrypted traffic. Enterprises might proscribe
service providers that either allow policy violations or cause a
competitive disadvantage.  Campus networks could desire to prevent
games from clogging up the available bandwidth.  From the point of
view of a network provider, better classification and understanding of
traffic could allow better traffic shaping. Entities interested in
policy enforcement could also derive more meta data by such analysis.

We shall use the term \emph{identification problem} to mean the act of
identifing traffic behind encrypted channels.  There are several
published techniques resulting from many years of research in order to
classify traffic -- most of them rely on machine learning.


We are releasing a simple tool that we call {\tt pacumen}, that relies
on simple, easily collectible features in order to solve the
\emph{identification problem}. This tool is intended to be be easy to
use, and when compared to the academic approaches, need smaller amount
of data for training and lesser user intervention for it to
function. Additionally, as this tool is released as an open source
project on {\tt github}, we expect users to extend this tool and
contribute to its growth.  Our main purpose in this exercise to gain
insight and understanding to the phenomenon of hiding traffic behind
encrypted tunnels and the process of applying machine learning
techniques in an attempt to uncover them.  This work should both lead
to better techniques on both sides of this hide-and-seek game. Lastly,
we present some experiments and their results to validate {\tt
  pacumen}.
