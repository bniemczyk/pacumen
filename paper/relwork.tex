\section{Related Work/Prior Research}
Due to the fundamental nature of the \emph{identification problem}
there has been much work over a long period of time
that has explored the identification of applications/traffic
types over encrypted channels or at least in a payload agnostic manner. 
Table~\ref{tab1} illustrates a selection of this work. 

We cite two attempts at comparing various machine learning algorithms
in conjunction with several features to solve the \emph{identification 
problem}. 
Lim \emph{et al} \cite{lim2000comparison} compare twenty-two decision
tree, nine statistical, and two neural network algorithms on
thirty-two datasets with respect to classification accuracy, training
time, and (in the case of trees) number of leaf nodes.  They measure
classification accuracy by mean error rate and mean rank of error
rate.  They state that most of these algorithms perform similarly with
a statistical, spline-based, algorithm called Polyclass working
slightly better than other algorithms.  Another statistical algorithm,
logistic regression, is second with respect to the two accuracy
criteria.  They find that the most accurate decision tree algorithm is
Quest with linear splits, which ranks fourth and fifth, respectively.
Although spline-based statistical algorithms tend to have good
accuracy, they also require relatively long training times. Polyclass,
for example, is third last in terms of median training time. It often
requires hours of training compared to seconds for other
algorithms. The Quest and logistic regression algorithms are
substantially faster. Among decision tree algorithms with univariate
splits, C4.5, Ind-Cart, and Quest have the best combinations of error
rate and speed. But C4.5 tends to produce trees with comparitively a
large number of leaves.
Alshamarri \emph{et al} \cite{alshammari2009machine} compare AdaBoost,
Support Vector Machines, Naive Bayesian, RIPPER and C4.5.  They find
that C4.5 is the best of these.



Some techniques such as the one by Bernaille \emph{et
  al}\cite{bernaille2006traffic} require very few features to be
collected -- only the first five packets are used. The number five was
arrived by trying different number of packets, and different numbers of 
clusters and finding the pair $\langle 5,50 \rangle$ to yield the best
results.


 Karagiannis
\emph{et al} \cite{karagiannis2005blinc} enhance their classification
mechanism with some social, functional and application level
information, and obtain good accuracy.  

Roughan \emph{et al}
\cite{roughan2004class} perform classification for the purpose of
quality of service(QOS) maintenance.


\begin{table}[htbp]
\begin{small}
\begin{tabular}{||p{30mm}|p{35mm}|p{25mm}|p{25 mm}||}
\hline
Authors & ML Technique & Data, Quantity, Features & Accuracy \\
\hline
\hline
Williams et al \cite{williams2006preliminary}  & 
(i) Baysesian with discretized featured (ii) C4.5 (iii) Naive Bayes (iv) Naive Bayes Tree.   Feature reduction using correlation and consistency. 
Filter vs. wrapper models.  Sampling to limit number of flows. 

 &auckland-vi-2001061{1,2}  leipzig-ii-20030221, nzix-ii20000706 4000 Flows per application class ; 22 Features & $> 80 \%$ \\
\hline
%------------------------------------------------------------------
McGregor \emph{et al} \cite{mcgregor2004flow} &  Expectation Maximization(EM)   & packet length, inter-arrival time and flow duration. & NA, unsupervised\\
%--------------------------------------------------------------------------------------
%\hline
%\cite{dunnigan2000flow} & PCA to detect consistent statistical patterns & &  \\
%-----------------------------------------------------
\hline
Zander \emph{et al}\cite{zander2005automated} &  Greedy Forward Feature Search and EM   & subset of Table~\ref{flfeatures} &  average of $86.5\%$\\
%-----------------------------------------------------
\hline
Roughan \emph{et al} \cite{roughan2004class} &  Nearest Neighbor (NN) and linear discriminate analysis (LDA) into interactive or  transactional& packet, flow, connection, intra-flow, multi-flow & 90s \\
%-----------------------------------------------------
\hline
Moore \emph{et al} \cite{moore2005internet}  & Naive Bayes Classifier, Correlation based  selection of stronger features  &  248 flow features -- packet length, inter-arrival times & $> 65 \%$ per flow. $70 - 97 \%$ with Kernel density estimation and  and $> 90\%$ with FCBF Filtering  \\
%---------------------------------
\hline
Karagiannis {\em et al} \cite{karagiannis2005blinc}&graph-based& social, functional, application& $> 95 \%$\\
\hline
Bernaille {\em et al} \cite{bernaille2006traffic}&  K-means clusterting & First 5 packets&$> 80 \%$  \\
\hline
Lim \emph{et al}\cite{lim2000comparison} &  33 algorithms &  32 data sets &\\
\hline
Alshammari, Zincir-Heywood \cite{alshammari2009machine}& 5 algorithms & ssh v.s skype features in Table~\ref{flfeatures} on old standard data &  $>90\%$ \\
\hline
Zhang {\em et al} \cite{zhang2013robust}& k-NN, Bayesian, random forest & number of packets, bytes transferred, packet size, inter-packet time & variable\\
\hline
Branch {\em et al}\cite{branch2009rapid}& C4.5 & Skype identification.  Packet lengths, inter arrival times, large v/s small packets& $>98\%$\\
\hline
\end{tabular}
\end{small}
\caption{\label{tab1} Related Machine Learning Approaches Compared}
\end{table}




\begin{table}[htbp]
\begin{tabular}{|l|l|}
\hline
 Protocol &   Duration of the flow\\
\hline
\# Packets in forward direction & \# Bytes in forward direction\\
\# Packets in backward direction & \# Bytes in backward direction\\
Min forward inter-arrival time  & Min backward inter-arrival time\\
Std deviation of forward interarrival
times &
Std deviation of backward interarrival
times\\
Mean forward inter-arrival time & Mean backward inter-arrival time\\
Max forward inter-arrival time &  Max backward inter-arrival time\\
Min forward packet length & Min backward packet length\\
Max forward packet length & Max backward packet length\\
Std deviation of forward packet
length &
Std deviation of backward packet
length\\
Mean backward packet length &Mean forward packet length\\
\hline
\end{tabular}
\caption{\label{flfeatures}Typical feature set for a ML algorithm}
\end{table}

The results of Table~\ref{tab1} would perhaps suggest that it is easy
to get accuracies of greater than $80\%$ while solving the identification
problem using machine learning techniques.  While we have had similar
successes in attempting to identify Skype against non-Skype, we have
not had similar successes with other classification attempts over
browser-based applications such as gmail.

Absorbing the work that went behind all these papers gave us much better insight and intuition into the nature of the identification problem and the techniques to solve it than if we were to attempt solving it from scratch. 
